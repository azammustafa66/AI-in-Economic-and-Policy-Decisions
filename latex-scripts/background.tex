\chapter{Background}
\justifying
\label{sec:background}

This section provides an overview of the key concepts and techniques used in this work, focusing on the comparison between econometric models and machine learning techniques in the context of economic forecasting and policy design. We will cover the following topics:

\begin{itemize}
    \item \textbf{Econometric Models:} A brief overview of traditional econometric models used in economic analysis and their assumptions, including models like linear regression, ARIMA, and Vector Autoregression (VAR).
    \item \textbf{Machine Learning Techniques:} An introduction to machine learning techniques, such as tree-based ensemble methods (e.g., Random Forest, Gradient Boosting) and neural networks, and how they are being used to enhance prediction accuracy in economic forecasting.
    \item \textbf{Model Evaluation:} A discussion on model evaluation metrics, emphasizing how they are used to assess the performance of econometric and machine learning models.
    \item \textbf{Comparison and Hybrid Approaches:} An exploration of how combining econometric models and machine learning can improve predictive performance in the application of AI to economic forecasting and policy design.
\end{itemize}

\section{Econometric Models}
\label{subsec:econometric_models}
Econometric models, such as linear regression and various time series models (e.g., ARIMA, VAR), have been traditionally used in economic forecasting. These models assume specific underlying data generating processes, often linear relationships, and rely on statistical methods for parameter estimation. While widely used, they can struggle with capturing complex, nonlinear relationships and effectively handling large datasets. This work will compare these models with machine learning techniques to assess their performance in economic prediction tasks relevant to AI-driven forecasting and policy design.

\section{Machine Learning Techniques}
\label{subsec:machine_learning_techniques}
Machine learning techniques, including tree-based ensemble methods like Random Forest and Gradient Boosting, as well as Neural Networks, are increasingly being applied in economics due to their capability to model complex, nonlinear relationships and process large datasets. In this work, we explore how these techniques can offer improvements over traditional econometric models, particularly in terms of predictive accuracy and their ability to leverage modern economic data for enhanced AI-driven insights.

\section{Model Evaluation}
\label{subsec:model_evaluation}
To evaluate model performance in both econometric and machine learning paradigms, we focus on metrics such as Mean Absolute Error (MAE), Mean Squared Error (MSE), and R-squared. These metrics will allow us to quantitatively assess the predictive accuracy and effectiveness of each approach in capturing the complexities inherent in economic data relevant to forecasting and policy formulation within an AI framework.

\section{Comparison and Hybrid Approaches}
\label{subsec:hybrid_approaches}
We also explore hybrid modeling strategies that aim to integrate the strengths of traditional econometric models with the predictive power of machine learning techniques. These hybrid approaches seek to leverage the interpretability often associated with econometrics while harnessing the flexibility and accuracy of machine learning, potentially offering more robust and insightful solutions for AI applications in economic forecasting and policy design.