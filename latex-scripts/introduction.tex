\chapter{\uppercase{Introduction}}
\justifying
\label{chap:introduction}

Artificial Intelligence (AI) is a branch of computer science focused on creating systems capable of performing tasks that typically require human intelligence. These tasks include learning, reasoning, decision-making, and understanding natural language. With the explosion of data and computational power in recent decades, AI has evolved from a theoretical concept into a transformative technology, influencing various aspects of modern life—including economics and policy design. \citep{googleai-overview}

\section{Brief History of AI}
\label{sec:history-of-ai}

The term \textit{Artificial Intelligence} was coined in 1956 by John McCarthy during the Dartmouth Conference, marking the formal beginning of AI as a field of study. Early AI programs focused on symbolic reasoning and problem-solving—such as Christopher Strachey’s checkers-playing program (1951) and ELIZA (1966), a text-based simulation of a psychotherapist.

The 1970s and 1980s witnessed the rise of expert systems like MYCIN, designed to diagnose bacterial infections using rule-based logic. However, this period was also marked by growing disillusionment due to technological limitations, leading to the "AI winter"—a sharp decline in research funding and interest.

A resurgence began in the late 1980s and 1990s with the emergence of machine learning and neural networks. Landmark achievements such as IBM’s \textit{Deep Blue} defeating world chess champion Garry Kasparov (1997) demonstrated AI’s increasing capabilities.

The 2000s and 2010s ushered in a new era of AI, driven by big data, cloud computing, and deep learning. Breakthroughs in natural language processing, computer vision, and reinforcement learning—highlighted by \textit{AlphaGo’s} victory over Go champion Lee Sedol (2016)—signaled a significant leap in AI sophistication. More recently, advances such as transformer models and generative adversarial networks (GANs) have further extended AI’s reach into language generation, image synthesis, and decision support systems. \citep{tableauhistory}

\section{Applications of AI}
\label{sec:applications-of-ai}

AI technologies are now integral to a wide array of industries:

\begin{itemize}
    \item \textbf{Healthcare:} Diagnostic imaging, predictive analytics, and personalized medicine.
    \item \textbf{Finance:} Fraud detection, algorithmic trading, and credit scoring.
    \item \textbf{Transportation:} Autonomous vehicles and smart traffic systems.
    \item \textbf{Retail:} Customer behavior analytics and recommendation engines.
    \item \textbf{Manufacturing:} Predictive maintenance, quality control, and robotics.
    \item \textbf{Entertainment:} Content curation, gaming AI, and virtual assistants.
    \item \textbf{Agriculture:} Precision farming, yield prediction, and pest detection.
    \item \textbf{Cybersecurity:} Threat detection and automated response.
    \item \textbf{Smart Cities:} Resource management, public safety, and urban planning.
    \item \textbf{Natural Language Processing (NLP):} Language translation, sentiment analysis, and conversational AI.
    \item \textbf{Computer Vision:} Facial recognition, object detection, and video surveillance.\\
    \citep{googleai-applications}
\end{itemize}

These applications highlight AI’s expanding role in improving efficiency, productivity, and decision-making across sectors.

\section{Economic Forecasting}
\label{sec:economic-forecasting}

Economic forecasting involves predicting future macroeconomic conditions using historical data and statistical models. It plays a critical role in policy-making, business planning, and investment decisions. Common indicators include GDP, inflation, employment, interest rates, and consumer spending.

Forecasting approaches fall into two broad categories:

\begin{itemize}
    \item \textbf{Qualitative Methods:} Rely on expert judgment, surveys, and experience-based\\
          assessments—often useful when data is scarce or when dealing with unprecedented situations.
    \item \textbf{Quantitative Methods:} Use mathematical models and historical data to predict economic trends. This includes:
          \begin{itemize}
              \item \textit{Time Series Models}, such as ARIMA and SARIMA, which analyze past values to forecast future trends.
              \item \textit{Causal Models}, which use economic theory to model relationships between variables (e.g., how inflation affects interest rates).
          \end{itemize}
\end{itemize}

While widely used, these traditional models have notable limitations.

\section{Limitations of Traditional Forecasting Methods}
\label{sec:limitations-of-traditional-methods}

Despite their historical success, conventional forecasting techniques face increasing challenges in the modern economic landscape:

\begin{itemize}
    \item \textbf{Dependence on Historical Data:} These models struggle with structural changes and novel trends not reflected in past data.
    \item \textbf{Simplifying Assumptions:} Many rely on assumptions such as linearity or stationarity, which may not hold in real-world economic systems.
    \item \textbf{Human Bias:} Expert-driven processes can introduce subjectivity and cognitive bias.
    \item \textbf{Sensitivity to External Shocks:} Traditional models often fail to incorporate sudden geopolitical events, natural disasters, or unexpected policy changes—especially those originating from abroad. \citet{fastercapital2024}
\end{itemize}

These limitations underscore the need for more adaptive and data-driven approaches. In this context, AI and machine learning offer promising alternatives by enabling models to learn complex patterns, adapt to changing conditions, and process vast volumes of structured and unstructured data.