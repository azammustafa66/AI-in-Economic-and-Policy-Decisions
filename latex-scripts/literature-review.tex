\chapter{\uppercase{Literature Review}}
\justifying
\label{chap:literature-review}

Artificial Intelligence (AI) has gained increasing attention in the field of economic forecasting and policy design.
Several studies have explored the effectiveness of machine learning (ML) models compared to traditional econometric methods.
This section reviews recent literature on this topic, focusing on methodologies, findings, and limitations, and highlighting areas where further research is needed.

\section{Machine Learning vs. Traditional Econometric Models}
\label{sec:ml-vs-traditional-econometric-models}

Machine learning models have demonstrated superior performance in forecasting economic conditions compared to
classical statistical techniques. A study by \citet{HarishParuchuri} on the Italian economy (1995–2015) employed models such as the
Nonlinear Autoregressive model with Exogenous Inputs (NARX), showing that ML algorithms could reliably predict
recessions with a lead time of one to two quarters. Unlike traditional methods, these models performed well even
with unadjusted, raw economic parameters. The findings reinforce the growing consensus that machine learning is an
effective and adaptable tool for economic forecasting, offering higher accuracy and earlier warning signs of downturns.

\citet{Varian2014-ez} emphasizes that most traditional econometric models are designed for causal inference, while ML models prioritize prediction accuracy.
This distinction is particularly important in economic forecasting, where the primary objective is to anticipate future outcomes rather than establish causal relationships.

\section{Machine Learning in Financial Forecasting}
\label{sec:ml-in-financial-forecasting}

Beyond macroeconomic forecasting, machine learning has also shown strong results in the domain of financial forecasting,
particularly in predicting stock prices and market behavior. \citet{Hsu2016-in} compared the performance of traditional econometric models
with ML techniques such as Support Vector Machines (SVM) and Artificial Neural Networks (ANN).
Their study reported that machine learning models outperformed traditional methods, achieving over 80\% accuracy in stock price prediction.

Similarly, \citet{Chernysh2024-vg} found that using ML models, particularly Random Forest and Gradient Boosting, was more effective than the ARIMA (AutoRegressive Integrated Moving Average) model for financial prediction tasks.
The authors also stressed the significance of feature selection and data preprocessing in enhancing the predictive power of ML models.

\section{Machine Learning in Macroeconomic Forecasting}
\label{sec:ml-in-macroeconomic-forecasting}

In macroeconomic forecasting, machine learning has been applied to predict indicators such as GDP growth, inflation, and unemployment rates.
Hybrid models that combine ML techniques with traditional econometric methods have shown promise in improving accuracy.
For instance, \citet{SaurabhGhosh-AbhishekRanjan} and \citet{Yusuf2025-pc} proposed hybrid frameworks that outperformed standalone ML or econometric approaches.
Their models closely matched actual economic data, suggesting the advantage of leveraging both data-driven and theory-based methods.

These studies collectively suggest that machine learning offers a significant advantage in economic forecasting, particularly when integrated with traditional econometric approaches.
However, gaps remain in the application of newer ML models, their interpretability, and their relevance for real-time policy analysis.

\section{Summary of Reviewed Literature}
\label{sec:summary-of-reviewed-literature}


\begin{table}[H]
    \centering
    \caption{Summary of Key Literature on ML in Economic Forecasting}
    \renewcommand{\arraystretch}{1.3}
    \begin{tabularx}{\textwidth}{|X|X|X|X|}
        \hline
        \textbf{Study}          & \textbf{Focus Area}          & \textbf{Model(s) Used}         & \textbf{Key Findings}                                              \\
        \hline
        \citet{HarishParuchuri} & Recession Prediction (Italy) & NARX                           & ML outperformed traditional models; provided 1–2 quarter lead time \\
        \hline
        \citet{Varian2014-ez}   & ML vs Econometrics           & General ML comparison          & ML excels in prediction; econometrics suited for causality         \\
        \hline
        \citet{Hsu2016-in}      & Financial Forecasting        & SVM, ANN                       & ML achieved over 80\% accuracy in stock price prediction           \\
        \hline
        \citet{Chernysh2024-vg} & Financial Forecasting        & RF, Gradient Boosting vs ARIMA & ML more effective; emphasized feature selection and preprocessing  \\
        \hline
        \makecell[tl]{\citeauthor{SaurabhGhosh-AbhishekRanjan},                                                                                                      \\ \citeauthor{Yusuf2025-pc}} & Macroeconomic Forecasting & Hybrid Models & Hybrid models outperformed individual ML or econometric models \\
        \hline
    \end{tabularx}
\end{table}

\section{Research Gap and Direction}
\label{sec:research-gap-and-direction}

While machine learning models have shown significant promise in economic forecasting, limitations such as model interpretability, applicability to real-time policy design, and use in emerging economies remain underexplored.
Additionally, newer ML models like XGBoost, NGBoost, TabNet, and Temporal Fusion Transformers have not been fully assessed in this domain.

This dissertation aims to build upon the reviewed literature by developing a hybrid forecasting model that integrates traditional econometric techniques with modern ML methods.
The model will be evaluated against established ML techniques and classical models to assess improvements in accuracy, robustness, and policy relevance.
Furthermore, newer ML methods such as XGBoost and NGBoost will be tested for their predictive capabilities on economic indicators like GDP growth, inflation, and unemployment, providing a more comprehensive understanding of their utility in economic forecasting.